\documentclass[11pt,a4paper]{scrartcl}
\usepackage[T1]{fontenc}
\usepackage[utf8]{inputenc}
\usepackage[ngerman]{babel}
\usepackage{microtype}
\usepackage{lmodern}
\usepackage{amsmath}
\usepackage{amsfonts}
\usepackage{amssymb}
\usepackage{enumerate}
\usepackage{graphicx}


\def\ojoin{\setbox0=\hbox{$\bowtie$}%
  \rule[-.02ex]{.25em}{.4pt}\llap{\rule[\ht0]{.25em}{.4pt}}}
\def\leftouterjoin{\mathbin{\ojoin\mkern-5.8mu\bowtie}}
\def\rightouterjoin{\mathbin{\bowtie\mkern-5.8mu\ojoin}}
\def\fullouterjoin{\mathbin{\ojoin\mkern-5.8mu\bowtie\mkern-5.8mu\ojoin}}

\begin{document}

\author{Johannes Merkle\\Ralf Vogler}
\title{Query Optimization}
\subtitle{2. Exercise}

\maketitle

\section*{Exercise 1}

\subsubsection*{Part 1}
The expressions are not equal. Example:\\

\begin{minipage}{.2\textwidth}
\begin{tabular}[t]{|c|c|}
\hline
  \multicolumn{2}{|c|}{R} \\
  \hline
  A & B\\ \hline \hline
  $a_1$ & $b_1$\\
  $a_1$ & $b_2$\\
  $a_2$ & $b_1$\\
  \hline
\end{tabular}
\end{minipage}
\begin{minipage}{.2\textwidth}
\begin{tabular}[t]{|c|c|}
\hline
  \multicolumn{2}{|c|}{S} \\
  \hline
  A & B\\ \hline \hline
  $a_1$ & $b_1$\\
  \hline  
\end{tabular}
\end{minipage}\\

 results in\\

\begin{minipage}{.2\textwidth}
\begin{tabular}{|c|}
\hline
  $\prod_A(R-S)$\\ \hline
  A\\ \hline \hline
  $a_1$\\ \hline
  $a_2$\\
  \hline  
\end{tabular}
\end{minipage}
\begin{minipage}{.2\textwidth}
\begin{tabular}{|c|}
\hline
  $\prod_A(R)-\prod_A(S)$s\\ \hline
  A\\ \hline \hline
  $a_2$\\
  \hline  
\end{tabular}
\end{minipage}\\


\subsubsection*{Part 2}

\begin{minipage}{.2\textwidth}
\begin{tabular}{|c|c|}
\hline
  \multicolumn{2}{|c|}{R} \\
  \hline
  $A_R$ & $A_J$\\ \hline \hline
  1 & x\\
  2 & a\\
  \hline
\end{tabular}
\end{minipage}
\begin{minipage}{.2\textwidth}
\begin{tabular}{|c|c|}
\hline
  \multicolumn{2}{|c|}{S} \\
  \hline
  $A_S$ & $A_J$\\ \hline \hline
  3 & a\\
  \hline
\end{tabular}
\end{minipage}
\begin{minipage}{.2\textwidth}
\begin{tabular}{|c|c|}
\hline
  \multicolumn{2}{|c|}{T} \\
  \hline
  $A_T$ & $A_J$\\ \hline \hline
  4 & x\\
  5 & a\\
  \hline
\end{tabular}
\end{minipage}\\

results in:\\

\begin{minipage}{.3\textwidth}
\begin{tabular}{|c|c|c|c|}
\hline
  \multicolumn{4}{|c|}{$(R \leftouterjoin S)\leftouterjoin T$} \\
  \hline
  $A_R$ & $A_J$ & $A_S$ & $A_T$\\ \hline \hline
  1 & x & & 4\\
  2 & a & 3 & 5\\
  \hline
\end{tabular}
\end{minipage}
\begin{minipage}{.3\textwidth}
\begin{tabular}{|c|c|c|c|}
\hline
  \multicolumn{4}{|c|}{$R \leftouterjoin (S\leftouterjoin T)$} \\
  \hline
$A_R$ & $A_J$ & $A_S$ & $A_T$\\ \hline \hline
  1 & x & & \\
  2 & a & 3 & 5\\
  \hline
\end{tabular}
\end{minipage}\\


\section*{Exercise 2}

\subsubsection*{Part 1}

If $R1.x$ is a key, we can have at most 1 entry, as keys are unique. In this case, the selectivity is $\frac{1}{|R1|}$. If it is not a key, a way to estimate the selectivity is to assume uniform distribution of values of the domain and therefore the selectivity can be estimated as $\frac{|R1.x|}{|R1|}$ where $|R1.x|$ denotes the number of values of the domain.

\subsubsection*{Part 2}

Given an estimation for the selectivity of $\sigma_{R1.x=c}$, we can estimate the selectivity of $\Join_{R1.x=R2.y}$ as $selectivity(\sigma_{R1.x=c})*selectivity(\sigma_{R1.x=c})$, though this would not be very accurate.

\section*{Exercise 3}

see files in folder \textit{tinydb}

\end{document}
