\documentclass[11pt,a4paper]{scrartcl}
\usepackage[T1]{fontenc}
\usepackage{microtype}
\usepackage{lmodern}
\usepackage{amsmath}
\usepackage{amsfonts}
\usepackage{amssymb}
\usepackage{enumerate}
\usepackage{graphicx}
\usepackage{float}

\usepackage{listings}
\usepackage{color}
%% http://stackoverflow.com/questions/741985/latex-source-code-listing-like-in-professional-books
\usepackage{courier}
\definecolor{light-gray}{gray}{0.95}
\lstset{
  % language=C,
  basicstyle=\small\sffamily,
%   basicstyle=\small\ttfamily,
  numbers=left,
  numberstyle=\tiny,
  frame=tb,
%  columns=fullflexible,
%  showstringspaces=false,
	backgroundcolor=\color{light-gray},
	linewidth=\linewidth,       % Zeilenbreite
	breaklines=true,             % Zeileumbruch
	breakatwhitespace=false, %Umbruch an Leerzeichen
  tabsize=2,
  extendedchars=true,
  xleftmargin=17pt,
  framexleftmargin=17pt,
  abovecaptionskip=7pt,
%   frameround=tttt,
}

\def\ojoin{\setbox0=\hbox{$\bowtie$}%
  \rule[-.02ex]{.25em}{.4pt}\llap{\rule[\ht0]{.25em}{.4pt}}}
\def\leftouterjoin{\mathbin{\ojoin\mkern-5.8mu\bowtie}}
\def\rightouterjoin{\mathbin{\bowtie\mkern-5.8mu\ojoin}}
\def\fullouterjoin{\mathbin{\ojoin\mkern-5.8mu\bowtie\mkern-5.8mu\ojoin}}

\begin{document}

\author{Johannes Merkle\\Ralf Vogler}
\title{Query Optimization}
\subtitle{10. Exercise}

\maketitle

\section*{Exercise 1 - Bandwidth and seek time}
See folder \verb|java| for the code. \verb|results.txt| contains some test results. Drive \verb|C| is a SSD and drive \verb|G| is a NDAS connected over Gigabit-Ethernet. Results using other tools are shown in the images in this folder.

%\lstinputlisting[caption={Some test results},label=lst:results]{results.txt}


\section*{Exercise 2 - Break even point}
SSD: Assuming 170 MB/s bandwidth and 300 $\mu$s seek time:
\begin{align*}
t_{page} = 8KB / (170 MB/s) &= 45.9559 \mu s\\
s * t_{page} &\stackrel{!}{=} x * s * 300\mu s\\
x &= 0.1532 \approx 15\%
\end{align*}
Internal 7200rpm HDD: Assuming 80 MB/s bandwidth and 12 ms seek time:
\begin{align*}
t_{page} = 8KB / (80 MB/s) &= 97.6563 \mu s\\
s * t_{page} &\stackrel{!}{=} x * s * 12ms\\
x &= 0.0081 \approx 1\%
\end{align*}
where s is the total number of pages of the relation and x is the selectivity ratio.
If the query selects more than x\% of the relation, then it will be faster to do a scan instead of an index lookup.

\end{document}
